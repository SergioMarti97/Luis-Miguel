% Options for packages loaded elsewhere
\PassOptionsToPackage{unicode}{hyperref}
\PassOptionsToPackage{hyphens}{url}
%
\documentclass[
]{article}
\usepackage{amsmath,amssymb}
\usepackage{lmodern}
\usepackage{iftex}
\ifPDFTeX
  \usepackage[T1]{fontenc}
  \usepackage[utf8]{inputenc}
  \usepackage{textcomp} % provide euro and other symbols
\else % if luatex or xetex
  \usepackage{unicode-math}
  \defaultfontfeatures{Scale=MatchLowercase}
  \defaultfontfeatures[\rmfamily]{Ligatures=TeX,Scale=1}
\fi
% Use upquote if available, for straight quotes in verbatim environments
\IfFileExists{upquote.sty}{\usepackage{upquote}}{}
\IfFileExists{microtype.sty}{% use microtype if available
  \usepackage[]{microtype}
  \UseMicrotypeSet[protrusion]{basicmath} % disable protrusion for tt fonts
}{}
\makeatletter
\@ifundefined{KOMAClassName}{% if non-KOMA class
  \IfFileExists{parskip.sty}{%
    \usepackage{parskip}
  }{% else
    \setlength{\parindent}{0pt}
    \setlength{\parskip}{6pt plus 2pt minus 1pt}}
}{% if KOMA class
  \KOMAoptions{parskip=half}}
\makeatother
\usepackage{xcolor}
\IfFileExists{xurl.sty}{\usepackage{xurl}}{} % add URL line breaks if available
\IfFileExists{bookmark.sty}{\usepackage{bookmark}}{\usepackage{hyperref}}
\hypersetup{
  pdftitle={feature selection},
  pdfauthor={Nuria \& Sergio},
  hidelinks,
  pdfcreator={LaTeX via pandoc}}
\urlstyle{same} % disable monospaced font for URLs
\usepackage[margin=1in]{geometry}
\usepackage{color}
\usepackage{fancyvrb}
\newcommand{\VerbBar}{|}
\newcommand{\VERB}{\Verb[commandchars=\\\{\}]}
\DefineVerbatimEnvironment{Highlighting}{Verbatim}{commandchars=\\\{\}}
% Add ',fontsize=\small' for more characters per line
\usepackage{framed}
\definecolor{shadecolor}{RGB}{248,248,248}
\newenvironment{Shaded}{\begin{snugshade}}{\end{snugshade}}
\newcommand{\AlertTok}[1]{\textcolor[rgb]{0.94,0.16,0.16}{#1}}
\newcommand{\AnnotationTok}[1]{\textcolor[rgb]{0.56,0.35,0.01}{\textbf{\textit{#1}}}}
\newcommand{\AttributeTok}[1]{\textcolor[rgb]{0.77,0.63,0.00}{#1}}
\newcommand{\BaseNTok}[1]{\textcolor[rgb]{0.00,0.00,0.81}{#1}}
\newcommand{\BuiltInTok}[1]{#1}
\newcommand{\CharTok}[1]{\textcolor[rgb]{0.31,0.60,0.02}{#1}}
\newcommand{\CommentTok}[1]{\textcolor[rgb]{0.56,0.35,0.01}{\textit{#1}}}
\newcommand{\CommentVarTok}[1]{\textcolor[rgb]{0.56,0.35,0.01}{\textbf{\textit{#1}}}}
\newcommand{\ConstantTok}[1]{\textcolor[rgb]{0.00,0.00,0.00}{#1}}
\newcommand{\ControlFlowTok}[1]{\textcolor[rgb]{0.13,0.29,0.53}{\textbf{#1}}}
\newcommand{\DataTypeTok}[1]{\textcolor[rgb]{0.13,0.29,0.53}{#1}}
\newcommand{\DecValTok}[1]{\textcolor[rgb]{0.00,0.00,0.81}{#1}}
\newcommand{\DocumentationTok}[1]{\textcolor[rgb]{0.56,0.35,0.01}{\textbf{\textit{#1}}}}
\newcommand{\ErrorTok}[1]{\textcolor[rgb]{0.64,0.00,0.00}{\textbf{#1}}}
\newcommand{\ExtensionTok}[1]{#1}
\newcommand{\FloatTok}[1]{\textcolor[rgb]{0.00,0.00,0.81}{#1}}
\newcommand{\FunctionTok}[1]{\textcolor[rgb]{0.00,0.00,0.00}{#1}}
\newcommand{\ImportTok}[1]{#1}
\newcommand{\InformationTok}[1]{\textcolor[rgb]{0.56,0.35,0.01}{\textbf{\textit{#1}}}}
\newcommand{\KeywordTok}[1]{\textcolor[rgb]{0.13,0.29,0.53}{\textbf{#1}}}
\newcommand{\NormalTok}[1]{#1}
\newcommand{\OperatorTok}[1]{\textcolor[rgb]{0.81,0.36,0.00}{\textbf{#1}}}
\newcommand{\OtherTok}[1]{\textcolor[rgb]{0.56,0.35,0.01}{#1}}
\newcommand{\PreprocessorTok}[1]{\textcolor[rgb]{0.56,0.35,0.01}{\textit{#1}}}
\newcommand{\RegionMarkerTok}[1]{#1}
\newcommand{\SpecialCharTok}[1]{\textcolor[rgb]{0.00,0.00,0.00}{#1}}
\newcommand{\SpecialStringTok}[1]{\textcolor[rgb]{0.31,0.60,0.02}{#1}}
\newcommand{\StringTok}[1]{\textcolor[rgb]{0.31,0.60,0.02}{#1}}
\newcommand{\VariableTok}[1]{\textcolor[rgb]{0.00,0.00,0.00}{#1}}
\newcommand{\VerbatimStringTok}[1]{\textcolor[rgb]{0.31,0.60,0.02}{#1}}
\newcommand{\WarningTok}[1]{\textcolor[rgb]{0.56,0.35,0.01}{\textbf{\textit{#1}}}}
\usepackage{graphicx}
\makeatletter
\def\maxwidth{\ifdim\Gin@nat@width>\linewidth\linewidth\else\Gin@nat@width\fi}
\def\maxheight{\ifdim\Gin@nat@height>\textheight\textheight\else\Gin@nat@height\fi}
\makeatother
% Scale images if necessary, so that they will not overflow the page
% margins by default, and it is still possible to overwrite the defaults
% using explicit options in \includegraphics[width, height, ...]{}
\setkeys{Gin}{width=\maxwidth,height=\maxheight,keepaspectratio}
% Set default figure placement to htbp
\makeatletter
\def\fps@figure{htbp}
\makeatother
\setlength{\emergencystretch}{3em} % prevent overfull lines
\providecommand{\tightlist}{%
  \setlength{\itemsep}{0pt}\setlength{\parskip}{0pt}}
\setcounter{secnumdepth}{-\maxdimen} % remove section numbering
\ifLuaTeX
  \usepackage{selnolig}  % disable illegal ligatures
\fi

\title{feature selection}
\author{Nuria \& Sergio}
\date{2022-11-22}

\begin{document}
\maketitle

\hypertarget{seleccion-de-caracteristicas}{%
\section{Seleccion de
caracteristicas}\label{seleccion-de-caracteristicas}}

Los análisis para la reducción de la dimensionalidad, se dividen en dos
grupos: la extracción de características y la selección de
características. La selección de características se centra en
seleccionar o filtrar las variables más relevantes para el análisis
estadístico.

Es recomendable aplicar algún tipo de filtro de selección, por dos
motivos: reducir el número de características (features) facilita y
agiliza el análisis al trabajar con un volumen de datos menor, trabajar
con muchas variables puede probocar un sobreajuste o \emph{overfitting}.

\hypertarget{adquisiciuxf3n-de-datos}{%
\subsection{Adquisición de datos}\label{adquisiciuxf3n-de-datos}}

Primero, leemos los datos con los que se va a trabajar. Se va a trabajar
sobre los datos de Luis Miguel, en concreto, sobre los ``counts''
detectados para cada gen.

Los ``counts'' de cada gen, estan guardados como un objeto ``.RDS''
dentro del proyecto. Están en forma de matriz (large matrix) que es la
forma que devuelve tximport los datos.

\begin{Shaded}
\begin{Highlighting}[]
\CommentTok{\# Cargar countData como un dataframe}
\NormalTok{mCountData }\OtherTok{\textless{}{-}} \FunctionTok{readRDS}\NormalTok{(}\StringTok{"../countData.RDS"}\NormalTok{)}
\NormalTok{dfCountData }\OtherTok{\textless{}{-}} \FunctionTok{as.data.frame}\NormalTok{(mCountData)}
\FunctionTok{ncol}\NormalTok{(mCountData)}
\end{Highlighting}
\end{Shaded}

\begin{verbatim}
## [1] 28
\end{verbatim}

\begin{Shaded}
\begin{Highlighting}[]
\FunctionTok{nrow}\NormalTok{(mCountData)}
\end{Highlighting}
\end{Shaded}

\begin{verbatim}
## [1] 38244
\end{verbatim}

La matriz se encuentra organizada de la siguiente forma: las filas son
los genes y las columnas las muestras. Para poder trabajar de forma que
los genes sean variables, se debe de transponer la matriz (cambiar filas
por columnas).

\begin{Shaded}
\begin{Highlighting}[]
\CommentTok{\# Transponer la matriz}
\NormalTok{mTemp }\OtherTok{\textless{}{-}} \FunctionTok{t}\NormalTok{(mCountData)}
\NormalTok{dfCountDataT }\OtherTok{\textless{}{-}} \FunctionTok{as.data.frame}\NormalTok{(mTemp)}
\FunctionTok{rm}\NormalTok{(mTemp)}
\end{Highlighting}
\end{Shaded}

\hypertarget{filtrar-nulos-o-nan}{%
\subsection{Filtrar Nulos o Nan}\label{filtrar-nulos-o-nan}}

Una primera selección, es filtrar las variables en busca de Nulos o Nan
(Not a number).

\begin{Shaded}
\begin{Highlighting}[]
\CommentTok{\# Asegurarse que todos son numericos}
\FunctionTok{sum}\NormalTok{(}\SpecialCharTok{!}\FunctionTok{apply}\NormalTok{(dfCountDataT, }\DecValTok{2}\NormalTok{, is.numeric))}
\end{Highlighting}
\end{Shaded}

\begin{verbatim}
## [1] 0
\end{verbatim}

\begin{Shaded}
\begin{Highlighting}[]
\CommentTok{\# Asegurarse que todos son numericos}
\FunctionTok{sum}\NormalTok{(}\FunctionTok{apply}\NormalTok{(dfCountDataT, }\DecValTok{2}\NormalTok{, is.null))}
\end{Highlighting}
\end{Shaded}

\begin{verbatim}
## [1] 0
\end{verbatim}

Todos los valores del dataframe son numericos, y ninguno es nulo.

\hypertarget{filtro-de-la-varianza}{%
\subsection{Filtro de la varianza}\label{filtro-de-la-varianza}}

La varianza es un estimador de la información. Una variable con una
mayor varianza, significa que contiene una mayor información que una
variable con menos varianza. Una variable con varianza 0, significa que
no varia, por lo que no aporta nada de información, ya que significa que
en la muestra, esa variable siempre toma el mismo valor.

Se puede filtrar para eliminar las variables (genes) que no varían.
Además, se puede observar la distribución de la varianza en el
dataframe, por si se quiere aplicar un filtro de varianza más
restrictivo.

Los genes que no varían, se pueden guardar, ya que el hecho que no
resulten alterados también es relevante a nivel biológico.

\begin{Shaded}
\begin{Highlighting}[]
\CommentTok{\# Calcular la varianza para cada columna}
\NormalTok{lVar }\OtherTok{\textless{}{-}} \FunctionTok{lapply}\NormalTok{(dfCountDataT, var)}
\CommentTok{\# Redondeo a 3 decimales}
\NormalTok{lVar }\OtherTok{\textless{}{-}} \FunctionTok{round}\NormalTok{(}\FunctionTok{as.numeric}\NormalTok{(lVar), }\DecValTok{3}\NormalTok{)}

\CommentTok{\# Genero un nuevo dataframe, con los valores de varianza asociado a cada gen}
\NormalTok{dfVar }\OtherTok{\textless{}{-}} \FunctionTok{data.frame}\NormalTok{(}\AttributeTok{gene\_id =} \FunctionTok{colnames}\NormalTok{(dfCountDataT), }\AttributeTok{var =}\NormalTok{ lVar)}

\CommentTok{\# Se contar ahora los genes cuya varianza es cercana a 0}
\FunctionTok{nrow}\NormalTok{(dfVar[dfVar}\SpecialCharTok{$}\NormalTok{var }\SpecialCharTok{==} \DecValTok{0}\NormalTok{, ])}
\end{Highlighting}
\end{Shaded}

\begin{verbatim}
## [1] 5530
\end{verbatim}

Hay 5530 genes cuya varianza es 0 o cercana a 0 (con un error de 3
decimales). Estas variables no son interesantes para un análisis
estadístico, ya que no esta aportando inforación. Pero, si que son
interesantes para un análisis funcional (que genes son, que función
realizan).

\begin{Shaded}
\begin{Highlighting}[]
\CommentTok{\# Listar los genes que no varían}
\NormalTok{lLowVarGenes }\OtherTok{\textless{}{-}}\NormalTok{ dfVar[dfVar}\SpecialCharTok{$}\NormalTok{var }\SpecialCharTok{==} \DecValTok{0}\NormalTok{, ]}\SpecialCharTok{$}\NormalTok{gene\_id}

\CommentTok{\# Seleccionar y guardar los genes del dataframe con varianza cercana a 0}
\FunctionTok{saveRDS}\NormalTok{(dfCountDataT[,lLowVarGenes], }\StringTok{"./countDataLowVar.RDS"}\NormalTok{)}

\CommentTok{\# Eliminar del dataframe los genes con varianza cercana a 0}
\FunctionTok{library}\NormalTok{(tidyverse)}
\end{Highlighting}
\end{Shaded}

\begin{verbatim}
## Warning: package 'tidyverse' was built under R version 4.2.1
\end{verbatim}

\begin{verbatim}
## -- Attaching packages --------------------------------------- tidyverse 1.3.2 --
## v ggplot2 3.3.5      v purrr   0.3.4 
## v tibble  3.1.6      v dplyr   1.0.10
## v tidyr   1.2.0      v stringr 1.4.0 
## v readr   2.1.2      v forcats 0.5.1
\end{verbatim}

\begin{verbatim}
## Warning: package 'readr' was built under R version 4.2.1
\end{verbatim}

\begin{verbatim}
## Warning: package 'dplyr' was built under R version 4.2.1
\end{verbatim}

\begin{verbatim}
## -- Conflicts ------------------------------------------ tidyverse_conflicts() --
## x dplyr::filter() masks stats::filter()
## x dplyr::lag()    masks stats::lag()
\end{verbatim}

\begin{Shaded}
\begin{Highlighting}[]
\NormalTok{lHighVarGenes }\OtherTok{\textless{}{-}}\NormalTok{ dfVar[dfVar}\SpecialCharTok{$}\NormalTok{var }\SpecialCharTok{\textgreater{}} \DecValTok{0}\NormalTok{, ]}\SpecialCharTok{$}\NormalTok{gene\_id}
\NormalTok{dfCountDataHighVar }\OtherTok{\textless{}{-}}\NormalTok{ dfCountDataT }\SpecialCharTok{\%\textgreater{}\%} \FunctionTok{select}\NormalTok{(}\FunctionTok{all\_of}\NormalTok{(lHighVarGenes))}

\FunctionTok{rm}\NormalTok{(lLowVarGenes, lHighVarGenes)}
\end{Highlighting}
\end{Shaded}

Si se quiere aplicar un filtro más estricto, aplicando un límite
(threshold) de la varianza mayor que 0, sería interesante estudiar la
distribución del número de genes según su varianza.

\begin{Shaded}
\begin{Highlighting}[]
\CommentTok{\# Histograma número genes según varianza}
\FunctionTok{library}\NormalTok{(ggplot2)}

\CommentTok{\# Creamos el gráfico}
\CommentTok{\# @see "http://www.sthda.com/english/wiki/ggplot2{-}histogram{-}plot{-}quick{-}start{-}guide{-}r{-}software{-}and{-}data{-}visualization"}
\NormalTok{gVar }\OtherTok{\textless{}{-}} \FunctionTok{ggplot}\NormalTok{(}\FunctionTok{data.frame}\NormalTok{(}\AttributeTok{x =} \FunctionTok{log}\NormalTok{(lVar[ lVar }\SpecialCharTok{\textgreater{}} \DecValTok{0}\NormalTok{])), }\FunctionTok{aes}\NormalTok{(}\AttributeTok{x=}\NormalTok{x)) }\SpecialCharTok{+} 
  \FunctionTok{geom\_histogram}\NormalTok{(}\AttributeTok{bins =} \DecValTok{120}\NormalTok{, }\AttributeTok{color=}\StringTok{"black"}\NormalTok{, }\AttributeTok{fill=}\StringTok{"white"}\NormalTok{) }\SpecialCharTok{+}
  \FunctionTok{xlab}\NormalTok{(}\StringTok{"Logarítmo de la Varianza"}\NormalTok{) }\SpecialCharTok{+} 
  \FunctionTok{ylab}\NormalTok{(}\StringTok{"Número de genes"}\NormalTok{) }\SpecialCharTok{+}
  \FunctionTok{ggtitle}\NormalTok{(}\StringTok{"Histograma Genes según varianza"}\NormalTok{)}

\CommentTok{\# Mostramos el gráfico}
\NormalTok{gVar}
\end{Highlighting}
\end{Shaded}

\includegraphics{feature_selection_files/figure-latex/unnamed-chunk-6-1.pdf}

Esta gráfica, puede ayudar a marcar un \emph{threshold} (un límite) para
seleccionar los genes que más información aporten. La gráfica es un
histograma en el cual se esta representando: en el eje Y, el número de
genes; en el eje X el logarítmo de la varianza. Se ha aplicado el
logarítmo porque la varianza toma valores muy extremos.

\begin{Shaded}
\begin{Highlighting}[]
\CommentTok{\# Si aplicasemos un filtro de varianza \textgreater{} 1}
\NormalTok{threshold }\OtherTok{\textless{}{-}} \DecValTok{1}
\NormalTok{nGenesByThreshold }\OtherTok{\textless{}{-}} \FunctionTok{length}\NormalTok{(dfVar[dfVar}\SpecialCharTok{$}\NormalTok{var }\SpecialCharTok{\textgreater{}}\NormalTok{ threshold, ]}\SpecialCharTok{$}\NormalTok{gene\_id)}

\CommentTok{\# Lo gráficamos}
\NormalTok{gVar }\SpecialCharTok{+} 
  \FunctionTok{ggtitle}\NormalTok{(}\FunctionTok{paste0}\NormalTok{(}\FunctionTok{c}\NormalTok{(}\StringTok{"Filtro var \textgreater{}"}\NormalTok{, threshold, }\StringTok{":"}\NormalTok{, nGenesByThreshold, }\StringTok{"genes"}\NormalTok{), }\AttributeTok{collapse =} \StringTok{" "}\NormalTok{)) }\SpecialCharTok{+}
  \FunctionTok{geom\_vline}\NormalTok{(}\FunctionTok{aes}\NormalTok{(}\AttributeTok{xintercept=}\FunctionTok{log}\NormalTok{(threshold)), }\AttributeTok{color=}\StringTok{"blue"}\NormalTok{, }\AttributeTok{linetype=}\StringTok{"dashed"}\NormalTok{, }\AttributeTok{size=}\DecValTok{1}\NormalTok{)}
\end{Highlighting}
\end{Shaded}

\includegraphics{feature_selection_files/figure-latex/unnamed-chunk-7-1.pdf}

\begin{Shaded}
\begin{Highlighting}[]
\CommentTok{\# Si aplicasemos un filtro de varianza \textgreater{} 100}
\NormalTok{threshold }\OtherTok{\textless{}{-}} \DecValTok{1500}
\NormalTok{nGenesByThreshold }\OtherTok{\textless{}{-}} \FunctionTok{length}\NormalTok{(dfVar[dfVar}\SpecialCharTok{$}\NormalTok{var }\SpecialCharTok{\textgreater{}}\NormalTok{ threshold, ]}\SpecialCharTok{$}\NormalTok{gene\_id)}

\CommentTok{\# Lo gráficamos}
\NormalTok{gVar }\SpecialCharTok{+} 
  \FunctionTok{ggtitle}\NormalTok{(}\FunctionTok{paste0}\NormalTok{(}\FunctionTok{c}\NormalTok{(}\StringTok{"Filtro var \textgreater{}"}\NormalTok{, threshold, }\StringTok{":"}\NormalTok{, nGenesByThreshold, }\StringTok{"genes"}\NormalTok{), }\AttributeTok{collapse =} \StringTok{" "}\NormalTok{)) }\SpecialCharTok{+}
  \FunctionTok{geom\_vline}\NormalTok{(}\FunctionTok{aes}\NormalTok{(}\AttributeTok{xintercept=}\FunctionTok{log}\NormalTok{(threshold)), }\AttributeTok{color=}\StringTok{"blue"}\NormalTok{, }\AttributeTok{linetype=}\StringTok{"dashed"}\NormalTok{, }\AttributeTok{size=}\DecValTok{1}\NormalTok{)}
\end{Highlighting}
\end{Shaded}

\includegraphics{feature_selection_files/figure-latex/unnamed-chunk-7-2.pdf}

\begin{Shaded}
\begin{Highlighting}[]
\FunctionTok{rm}\NormalTok{(threshold, nGenesByThreshold)}
\end{Highlighting}
\end{Shaded}

Lo que intentan mostrar estos gráficos, es que aplicando un filtro
``varianza \textgreater{} threshold'', se deshechan todos los genes que
se encuentran a la izquierda de la gráfica, quedando lo genes de la
derecha.

Otra vía para aplicar un filtro según la varianza, es seleccionando un
número fijo de variables que tengan la mayor varianza. En otras
palabras, ordenar las variables de mayor a menor información, y escoger
solamente las 100 primeras. Este cribado, lo realiza la función para el
PCA de DESeq2, operando por defecto con las 500 primeras variables con
mayor varianza.

\hypertarget{filtro-de-caracteruxedsticas-correlacionadas}{%
\subsection{Filtro de características
correlacionadas}\label{filtro-de-caracteruxedsticas-correlacionadas}}

\end{document}
